% !TEX encoding = UTF-8 Unicode
\documentclass[12pt,a4paper,oneside]{article}
\setcounter{tocdepth}{3}
\usepackage{graphicx}
\usepackage{amssymb}
\usepackage{amsmath}
\usepackage{titling}
\usepackage[ngerman]{babel}
\usepackage[utf8]{inputenc}

\newcommand{\R}{\mathbb{R}}
\newcommand{\N}{\mathbb{N}}
\newcommand\pmat[1]{\ensuremath{\begin{pmatrix}#1\end{pmatrix}}}
\newcommand\set[2]{\ensuremath{\left\{#1\;\middle|\;#2\right\}}}

\newtheorem{definition}{Definition}

\title{Evolutionäre Punkt-Suche in der diskretisierten Ebene}
\author{Jonathan Tobias Haßel}
%\institute{Institut für Informatik\\Rheinische Friedrich-Wilhelms Universität Bonn}

\begin{document}

\begin{titlepage}
	\centering
	{\LARGE Institut für Informatik\\Rheinische Friedrich-Wilhelms Universität Bonn\\}
	\vspace{2cm}
	{\large Bachelor-Arbeit zur Erlangung des akademischen Grades\\
	\vspace{.5cm}
	Bachelor of Science (B.Sc.)\\
	\vspace{.5cm}
	im Studiengang Informatik\\}
	\vfill
	{\huge\thetitle\par}
	\vfill
	Vorgelegt von\\
	\vspace{1cm}
	\raggedright
	\theauthor\\
	Schnorrenbergstraße 2, 53229 Bonn\\
	3113185\\
	6. Semester\\
	\vspace{1cm}
	Themensteller: Priv.-Doz. Dr. Elmar Langetepe\\
    Zweitgutachter:  \\
    \vspace{.5cm}
    \centering
    \today
\end{titlepage}

\section{Abstract}
Die optimale Strategie, um einen Punkt in der Ebene ohne Hinweise zu finden, ist, in einer Spirale vom Ausgangspunkt aus die Fläche exhaustiv nach dem Punkt abzusuchen. Diese Arbeit beschäftigt sich mit der Frage, ob durch evolutionäres Lernen eine solche optimale Strategie erlernt werden kann.\\
Zuerst wird das Suchproblem auf einer Geraden betrachtet. Dabei wird festgestellt, ob ein entsprechender, evolutionärer Algorithmus die dazugehörige optimale Strategie, nämlich die Verdopplungsstrategie, erlernen kann.

\section{Vorwort}
Im Zuge einer Projektarbeit mit dem Thema "Computational Geometry" habe ich viel Erfahrung zur Hinweis-basierten Suche gesammelt. In dieser Situation wurden dem Suchenden Hinweise in verschiedenen Formen geliefert, die zwar den Suchraum einschränkten, die Entwicklungszeit für Gegenstrategien deutlich erhöhten. Diese gesteigerte Komplexität drängte mich schon im Projekt zu der Annahme, dass maschinelles Lernen sowohl auf Hinweis-basierten Szenarien als auch, wie in diesem Fall, ohne einschränkende Hinweise einen deutlichen Vorteil für die Entwicklungszeit bedeuten könnten.
Andererseits wurde mein Interesse für Evolutionäre Algorithmen im Zuge meines Studiums in den entsprechenden Vorlesungen geweckt, es lag also Nahe, dass das in meiner Arbeit behandelte Thema Aspekte evolutionären Lernens enthalten würde.

\tableofcontents
\newpage

\section{Evolutionäre Algorithmen}
Ein evolutionärer Algorithmus beschreibt einen Suchprozess, der, gepaart mit Optimierungs-Mechanismen, eine günstige Lösung für ein eingegebenes Problem finden soll. Dabei werden nach der darwinistischen Theorie 

Ein Suchraum über zu optimierende Parameter wird aufgespannt 

\section{Definitionen}

\begin{definition}[Diskreter Teilraum]
Sei $P$ als Anzahl aller verfügbaren diskreten Teilräume gegeben und sei $p\in(0,1,\dots,P-1)$ der eindeutige Index des jeweiligen diskreten Teilraums. Dann ist ein diskreter Teilraum die Teilmenge:
\begin{align}
\R^2 \supset U^P_p = \set{a x}{x=\pmat{\cos{\gamma}\\\sin{\gamma}},a\in\R_+,\gamma = \frac{p}{P} \cdot 2\pi}
\end{align}
\end{definition}
\begin{definition}[Diskreter Punkt]
Sei $P$ die Anzahl aller verfügbaren diskreten Teilräume.
Dann ist ein diskreter Punkt $A^P = (p,d)$ ein Tupel, dass aus einem ganzzahligen Positions-Wert $p\in\N$ und einer Distanz $d\in\R$ vom Ursprung besteht.
Weiterhin existieren Abbildungen $f_x$ und $f_y$, sodass $A^P$ in kartesische Koordinatendarstellung $(f_x,f_y)$ übertragbar ist, nämlich:
\begin{align}
\gamma = \frac{p(A^P)}{P} \cdot 2 \pi\\
f_x(A^P) = d(A^P)\cdot \sin{\gamma}\\
f_x(A^P) = d(A^P)\cdot \cos{\gamma}
\end{align}
\end{definition}
\begin{definition}[Individuum]
Sei $P$ die Anzahl aller verfügbaren diskreten Teilräume.
Dann ist ein Individuum ein $n$-Tupel $I^P=(A^P_1,\dots,A^P_n)$ von diskreten Punkten.\\
Übertragen auf die Problemstellung der Punkt-Suche wird der diskretisierte Suchraum in der Reihenfolge der Punkte im Tupel abgesucht.
\end{definition}
\begin{definition}[Punkt gefunden]
Sei $P$ die Anzahl aller verfügbaren diskreten Teilräume, ein diskreter Punkt $T^P$ der zu findende Punkt und der diskreten Punkt $S^P$ der suchende Punkt.
Dann ist die Suchfunktion folgendermaßen definiert:
\begin{align}
    finds(S^P,T^P)=(p(S^P) \iff p(T^P) \land d(S^P) \geq d(T^P))
\end{align}
\end{definition}

\begin{thebibliography}{8}
\bibitem{ref_article1}
Author, F.: Article title. Journal \textbf{2}(5), 99--110 (2016)
\end{thebibliography}
\end{document}
