\documentclass[runningheads]{llncs}
\setcounter{tocdepth}{3}

\usepackage{graphicx}
\usepackage{amssymb}
\usepackage{amsmath}
\usepackage{titling}
\usepackage[ngerman]{babel}

\newcommand{\R}{\mathbb{R}}
\newcommand{\N}{\mathbb{N}}
\newcommand\pmat[1]{\ensuremath{\begin{pmatrix}#1\end{pmatrix}}}
\newcommand\set[2]{\ensuremath{\left\{#1\;\middle|\;#2\right\}}}

\title{Evolution�re Punkt-Suche in der diskretisierten Ebene}
\author{Jonathan Tobias Ha�el}
\institute{Institut f�r Informatik\\Rheinische Friedrich-Wilhelms Universit�t Bonn}

\begin{document}

\begin{titlepage}
	\centering
	{\LARGE Institut f�r Informatik\\Rheinische Friedrich-Wilhelms Universit�t Bonn\\}
	\vspace{2cm}
	{\large Bachelor-Arbeit zur Erlangung des akademischen Grades\\
	\vspace{.5cm}
	Bachelor of Science (B.Sc.)\\
	\vspace{.5cm}
	im Studiengang Informatik\\}
	\vfill
	{\huge\thetitle\par}
	\vfill
	Vorgelegt von\\
	\vspace{1cm}
	\raggedright
	\theauthor\\
	Schnorrenbergstra�e 2, 53229 Bonn\\
	3113185\\
	6. Semester\\
	\vspace{1cm}
	Themensteller: Priv.-Doz. Dr. Elmar Langetepe\\
    Zweitgutachter:  \\
    \vspace{.5cm}
    \centering
    \today
\end{titlepage}

\section{Abstract}
Die optimale Strategie, um einen Punkt in der Ebene ohne Hinweise zu finden, ist, in einer Spirale vom Ausgangspunkt aus die Fl�che exhaustiv nach dem Punkt abzusuchen. Diese Arbeit besch�ftigt sich mit der Frage, ob durch evolution�res Lernen eine solche optimale Strategie erlernt werden kann.\\
Zuerst wird das Suchproblem auf einer Geraden betrachtet. Dabei wird festgestellt, ob ein entsprechender, evolution�rer Algorithmus die dazugeh�rige optimale Strategie, n�mlich die Verdopplungsstrategie, erlernen kann.

\section{Vorwort}
Im Zuge einer Projektarbeit mit dem Thema "Computational Geometry" habe ich viele Erfahrungen zur Hinweis-basierten Suche gesammelt. In dieser Konfiguration wurden dem Suchenden Hinweise in verschiedenen Formen geliefert, die zwar den Suchraum einschr�nkten, die Entwicklungszeit f�r Gegenstrategien deutlich erh�hten. Diese gesteigerte Komplexit�t dr�ngte mich schon im Projekt zu der Annahme, dass maschinelles Lernen sowohl auf Hinweis-basierten Szenarien als auch, wie in diesem Fall, ohne einschr�nkende Hinweise einen deutlichen Vorteil f�r die Entwicklungszeit bedeuten k�nnten.
Andererseits wurde mein Interesse f�r Evolution�re Algorithmen im Zuge meines Studiums in den entsprechenden Vorlesungen geweckt, es lag also Nahe, dass das in meiner Arbeit behandelte Thema evolution�re Aspekte enthalten w�rde.

\tableofcontents
\newpage

\section{Definitionen}

\begin{definition}[Diskreter Teilraum]
Sei $P$ die Anzahl aller verf�gbaren diskreten Teilr�ume und $p\in(0,1,\dots,P-1)$ der eindeutige Index des jeweiligen diskreten Teilraums. Dann ist ein diskreter Teilraum die Teilmenge:
\begin{align}
\R^2 \supset U^P_p = \set{a x}{x=\pmat{\cos{\gamma}\\\sin{\gamma}},a\in\R_+,\gamma = \frac{p}{P} \cdot 2\pi}
\end{align}
\end{definition}
\begin{definition}[Diskreter Punkt]
Sei $P$ die Anzahl aller verf�gbaren diskreten Teilr�ume.
Dann ist ein diskreter Punkt $A^P = (p,d)$ ein Tupel, dass aus einem ganzzahligen Positions-Wert $p\in\N$ und einer Distanz $d\in\R$ vom Ursprung besteht.
Weiterhin existieren Abbildungen $f_x$ und $f_y$, sodass $A^P$ in kartesische Koordinatendarstellung $(f_x,f_y)$ �bertragbar ist, n�mlich:
\begin{align}
\gamma = \frac{p(A^P)}{P} \cdot 2 \pi\\
f_x(A^P) = d(A^P)\cdot \sin{\gamma}\\
f_x(A^P) = d(A^P)\cdot \cos{\gamma}
\end{align}
\end{definition}
\begin{definition}[Individuum]
Sei $P$ die Anzahl aller verf�gbaren diskreten Teilr�ume.
Dann ist ein Individuum ein $n$-Tupel $I^P=(A^P_1,\dots,A^P_n)$ von diskreten Punkten.\\
�bertragen auf die Problemstellung der Punkt-Suche wird der diskretisierte Suchraum in der Reihenfolge der Punkte im Tupel abgesucht.
\end{definition}
\begin{definition}
Sei $P$ die Anzahl aller verf�gbaren diskreten Teilr�ume, ein diskreter Punkt $T^P$ der zu findende Punkt und der diskreten Punkt $S^P$ der suchende Punkt.
Dann ist die Suchfunktion folgenderma�en definiert:
\begin{align}
    finds(S^P,T^P)=(p(S^P) \iff p(T^P) \land d(S^P) \geq d(T^P))
\end{align}
\end{definition}

\begin{thebibliography}{8}
\bibitem{ref_article1}
Author, F.: Article title. Journal \textbf{2}(5), 99--110 (2016)

\bibitem{ref_lncs1}
Author, F., Author, S.: Title of a proceedings paper. In: Editor,
F., Editor, S. (eds.) CONFERENCE 2016, LNCS, vol. 9999, pp. 1--13.
Springer, Heidelberg (2016). \doi{10.10007/1234567890}

\bibitem{ref_book1}
Author, F., Author, S., Author, T.: Book title. 2nd edn. Publisher,
Location (1999)

\bibitem{ref_proc1}
Author, A.-B.: Contribution title. In: 9th International Proceedings
on Proceedings, pp. 1--2. Publisher, Location (2010)

\bibitem{ref_url1}
LNCS Homepage, \url{http://www.springer.com/lncs}. Last accessed 4
Oct 2017
\end{thebibliography}
\end{document}
